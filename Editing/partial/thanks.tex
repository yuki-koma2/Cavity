\chapter*{謝辞}
本研究を行うにあたり、北野先生をはじめとする北野研究室のみなさまには、
多大なるお力添えをいただきました。この場を借りて御礼申し上げます。
%本研究を行うにあたり、多くの人にお世話になりました。この場を借りてお礼申し上げます。
初めに、指導教官である北野晴久教授に感謝申し上げます。最初から自分勝手なことをしておりましたが、最後まで指導をしてくださりありがとうございました。
研究テーマも自分がシステムエンジニアとして就職することを考慮して、少しでも取り組みやすいようにシミュレーションのテーマを与えていただきました。とても感謝しております。
% 普段から適当に物事を行いがちな私ですが、輪講や実験を通して、物事の本質を見抜くために深く考えながら行動することの大切さを、先生の厳しくも熱心な指導から学ぶことができました。ありがとうございました。来年度も度々迷惑をお掛けするかもしれませんが、よろしくお願い致します。
次に、孫悦助教に感謝申し上げます。孫さんとは直接話すことは少なかったのですが、輪講などでいつも優しく手を差し伸べていただきました。
%試料の提供や電気化学処理用プログラムの作成、実験へのアドバイス等、様々な場面でお世話になりました。ありがとうございました。来年度も引き続きよろしくお願い致します。
修士2年の大沼遥さんに感謝申し上げます。
先輩1人という大変な立場でありながらも、質問したときには丁寧にわかりやすく教えていただきとても頼りになりました。
誰よりも早く来て誰よりも遅くまで研究室にいて、自身の研究に真剣に打ち込んでいた姿は敬服いたします。
%また、席が近いということもあり、よく他愛もない会話をしたり、時々食事に連れて行ってもらったりしました。来年度は大沼さんのような先輩になれるよう頑張ります。ありがとうございました。
同期の伊藤君、五島君、杉山君、田所君、寺元君、鳥海君、中村君、浦さん、峯君、堀川君、宮沢君、1年間お疲れ様でした。みんなには助けてもらってばかりでした。今度飲みに行きましょう。
%いつも全員とはいかなかったけど、時々呑みに行ったり、バスケをしたりすることもあって、楽しく充実した1年間になりました。ありがとう。田所君と宮沢君は、先輩がいないので大変な1年になると思うけど、来年度も一緒に頑張ろう。
最後に、ここまで支えてくれた家族に感謝申し上げます。学費等の工面、生活面でのサポートがあったからこそここまで頑張ることができました。
%帰省したときにいつも変わらず迎えてくれることをとても温かく感じています。ありがとうございます。これからもまだ負担を掛けると思いますがよろしくお願い致します。
