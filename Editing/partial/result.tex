\chapter{結果と考察}
共振周波数を変化させるためには、共振器の大きさを変化させる方法と、と内部に誘電体あるいは導体を挿入する方法がある。共振器自体の大きさを変化させて共振周波数を調整する方法では、全ての辺の長さの比を一定にしなければ、内部の電磁場分布が大きく変化してしまい、大きさの変化とともに試料設置位置も変動させる必要が出てくるが試料設置位置を動かすことは現実的ではない。また、全ての方向で大きさの比を保ちながら共振器の大きさを変化させるためにはかなり高度なギミックが必要となり、これも現実的ではない。そのため今回は内部に誘電体を挿入することによって、共振周波数を現象させるという方法で実現を目指す。


\section{誘電体挿入による影響}
まず、誘電体挿入による影響を調べるための検証を行った。

\vspace{10 mm}

\begin{figure}[h]
  \begin{center}
    \includegraphics[width=12cm]{./image/te101.png}
    \caption{TE101の電場分布}
    \label{fig:E-TE101}
  \end{center}
\end{figure}

上図のように基本モード$TE101$を持つ空洞共振器を用いて、誘電率9の誘電体(サファイア)を下図のように

\begin{enumerate}
  \item 共振器よりも小さいサイズで位置のみを変化させたモデル
  \item 共振器と同じサイズで徐々に挿入していくモデル
\end{enumerate}

の2つの場合で検証を行った。

\vspace{10 mm}

\begin{figure}[h]
  \begin{center}
    \includegraphics[width=8cm]{./image/pos.png}
    \caption{位置のみを変化させたモデル}
    \label{fig:potition}
  \end{center}
\end{figure}

\vspace{10 mm}

\begin{figure}[h]
  \begin{center}
    \includegraphics[width=11cm]{./image/length.png}
    \caption{誘電体を挿入していったモデル}
    \label{fig:length}
  \end{center}
\end{figure}

\subsection{結果}
結果を以下のグラフに示す。
この結果から、電場振幅の大きい場所に誘電体をおいた方が振幅が小さい場所におくよりも大きく共振周波数が変化すること。また、誘電体の位置だけを変化させる場合と挿入する場合では、変化の最大値は挿入していったモデルの方が大きいということがわかった。


\vspace{10 mm}

\begin{figure}[h]
  \begin{center}
    \includegraphics[width=12cm]{./image/plot1.jpg}
    \caption{誘電体による共振周波数の変化の検証}
    \label{fig:Cavity}
  \end{center}
\end{figure}


\subsection{考察}
これらの結果を受けて、共振周波数を大きく変化させるためには、共振器内部の「電場振幅の大きい場所」に誘電体を「挿入」すると良さそうである。

\section{設計したモデル}
この結果を受けて今回は下図(図4.5)の基本モードTE101 f=61GHz(固有値解析)となるような直方型空洞共振器を設計した。

\subsection{詳細設計}
具体的には、独立した3つの方向をx,y,z方向とよぶとする。
以下で使う長さの単位がmmである場合には省略して表記する。
基本となる空洞の大きさは(x,y,x)=(2.8,1.5,4.8)である。
共振器の底中央に大きさ(x,y,z)=(2.8,0.5,1.5)の基盤を設置した。(図4.7)
また、上部中央に共振周波数を調整するための誘電体として
大きさ(x,y,z)=(2.8,0.5,1.5)で挿入した。
挿入する誘電体は比誘電率9を持つサファイアを想定した。
誘電体、基盤として用いるサファイアは、北野研究室でよく使用されるサファイア基盤の厚さが5mmであるためそこから切り出しやすいように今回使用する厚みも5mmとした。
試料の位置は図4.5を参照。
また基盤と周波数調整誘電体の大きさが同じなのは試料設置位置にかかる電場を最大にするためである。
また過去に行われたcavity-QED実験\cite{cQED}を参考にして、外界との結合のため共振器上部に高さ1mm励振部を設置した。
外部との結合に用いる同軸ケーブルのサイズは外部導体の外径$φ=2.197mm$、内部誘電体層(テフロン)の外径$φ=1.676mm$、中心導体の外径$φ=0.511mm$とした。
また、空洞共振器の壁には完全導体(PEC)を使用し、内部は真空とした。

\vspace{10 mm}

\begin{figure}[h]
  \begin{center}
    \includegraphics[width=12cm]{./image/newmodel.png}
    \caption{設計したモデルの断面からの図}
    \label{fig:Cavity}
  \end{center}
\end{figure}

\vspace{10 mm}

\begin{figure}[h]
 \begin{minipage}{0.5\hsize}
  \begin{center}
   \includegraphics[width=70mm]{./image/model73.png}
  \end{center}
  \caption{設計したモデルの立体図}
  \label{fig:one}
 \end{minipage}
 \begin{minipage}{0.5\hsize}
  \begin{center}
   \includegraphics[width=70mm]{./image/基板資料あり.png}
  \end{center}
  \caption{基盤に試料を乗せた図}
  \label{fig:two}
 \end{minipage}
\end{figure}

実際に実験を行う際には、図4.7にあるように、
共振器の底に設置した誘電体の上には超伝導体の試料を載せる。

\subsection{基本性質}
この共振器の基本的な性質を示す。
以下それぞれ、左に固有値解析の結果、右に過渡解析の結果を示す。
過渡解析の結果はSパラメータを表示している。

\subsubsection{誘電体なし基盤なし}
まず誘電体も基盤も全く設置していない場合の解析結果は以下のようになっている。
過渡解析の結果、共振

\begin{figure}[h]
 \begin{minipage}{0.5\hsize}
  \begin{center}
   \includegraphics[width=70mm]{./image/空洞共振器.png}
  \end{center}
  \caption{一つめの図}
  \label{fig:one}
 \end{minipage}
 \begin{minipage}{0.5\hsize}
  \begin{center}
   \includegraphics[width=70mm]{./image/空洞共振器.png}
  \end{center}
  \caption{二つめの図}
  \label{fig:two}
 \end{minipage}
\end{figure}

\subsubsection{誘電体なし基盤あり}
\begin{figure}[h]
 \begin{minipage}{0.5\hsize}
  \begin{center}
   \includegraphics[width=70mm]{./image/空洞共振器.png}
  \end{center}
  \caption{一つめの図}
  \label{fig:one}
 \end{minipage}
 \begin{minipage}{0.5\hsize}
  \begin{center}
   \includegraphics[width=70mm]{./image/空洞共振器.png}
  \end{center}
  \caption{二つめの図}
  \label{fig:two}
 \end{minipage}
\end{figure}

\subsubsection{誘電体あり基盤あり}
\begin{figure}[h]
 \begin{minipage}{0.5\hsize}
  \begin{center}
   \includegraphics[width=70mm]{./image/空洞共振器.png}
  \end{center}
  \caption{一つめの図}
  \label{fig:one}
 \end{minipage}
 \begin{minipage}{0.5\hsize}
  \begin{center}
   \includegraphics[width=70mm]{./image/空洞共振器.png}
  \end{center}
  \caption{二つめの図}
  \label{fig:two}
 \end{minipage}
\end{figure}

\subsection{誘電体挿入時の共振周波数変化}

\vspace{10 mm}

\begin{figure}[h]
  \begin{center}
    \includegraphics[width=8cm]{./image/空洞共振器.png}
    \caption{方形空洞共振器}
    \label{fig:Cavity}
  \end{center}
\end{figure}

\subsection{考察}
