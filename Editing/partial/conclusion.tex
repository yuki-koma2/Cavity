\chapter{結論と今後の展望}
\section{結論}
今回の研究成果として、40〜52GHz程度で共振周波数が調整可能な、外界との結合、試料設置位置を含めたモデルの設計ができた。
% これでも十分実用に耐え得ると考えられるが、さらに精度をあげることも考えられる。
% また実際にこの実験を

% ------------
% ここもっと詳しくまとめる。具体的に。
% ------------

\section{今後の展望}

\subsection*{励振部分の調整}
励振部分の調整(例えば、長さの調整など)
を行い、より高いQ値が得られるようなモデルを考える。
% 今回作成したモデルの励振部に関しては、大きさを変えての検証は行わなかった。
% ここは
励振部外界(同軸ケーブル)の電磁場と共振器内部の電磁場の
結合に大きく関わる場所であり、
共振器の性能を決めるQ値(いわゆる負荷時のQ値)
に大きく影響すると考えられる。
% 共振する系には、その持続特性を表す量として、
% Q値と呼ばれる値が存在する。
一般にQ値は、単位時間に共振器に蓄えられる全エネルギーと損失エネルギーの比に
一致する。
c-QED実験の場合、
Q値が高いほど1マイクロ波光子の
% 存在時間
寿命が長くなるため、
原子(量子ビット)素相互作用する機会が増大するという意味で、
高いQ値を持った共振器を設計することは重要である。
% そのため、どの程度の大きさが最適なのかを別途調べる必要がある。

\subsection*{誘電体の挿入方法の検証}
今回作成したモデルでは約15GHz程度しか
共振周波数を調節できる幅がない。
固有ジョセフソン接合系の
今までの実験結果からこの幅でも
ある程度実用的と考えられるが、
調整可能領域は広い方が応用範囲がさらに広がる。
誘電体を2つに分割して挿入したり、
結合方法を変えて共振器自体の大きさを変え
たりするなどして、
共振周波数の調整
範囲を広げられないか検証する。

\subsection*{実際に試作する}
シュミレーターによる計算では、
% 必ず誤差が出てくる。
様々な近似や理想化が必ず含まれている。
そのため、実際にこのモデルを試作し共振器自体の特性を
実験から調べる必要がある。
