\chapter{結論と今後の展望}
\section{結論}
\section{今後の展望}

\subsection{励振部分の調整}
励振部分の調整を行い、より高いQ値が得られるようにモデルを考える。
今回作成したモデルの励振部に関しては、大きさを変えての検証は行わなかった。ここは外界(同軸ケーブル)の電磁場と共振器内部の電磁場の結合度に大きく関わる場所であり、共振器の性能を決めるQ値に大きく影響すると考えられる。共振する系には、その持続特性を表す量として、Q値と呼ばれる値が存在する。一般にQ値とは、共振器に蓄えられる全エネルギーと損失エネルギーの比に比例する。今回のc-QEDの実験では、Q値が高いほど1モード光子の存在時間が長くなるため、高いQ値を持った共振器を設計することは重要である。そのため、どの程度の大きさが最適なのかを別途調べる必要がある。

\subsection{誘電体の挿入方法の検証}
今回作成したモデルでは約15GHz程度しか調節できる幅がない。今までの結果からこの幅でも十分ではあると考えられるが、調整可能領域は広い方が応用範囲が広がるため、誘電体を2つに分割して挿入した場合や、結合方法を変えて共振器自体の大きさを変えることなどをして、範囲を広げられないか検証する。

\subsection{実際に試作する}
前述の通り、シュミレーターによる計算では必ず誤差が出てくる。そのため、実際にこのモデルを試作し共振器自体の特性を調べる必要がある。
