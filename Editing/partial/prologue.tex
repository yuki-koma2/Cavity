\chapter{序論}
\section{量子コンピューターの課題}
近年、量子コンピューターが話題に上がることが多くなってきた。量子コンピューターは超並列計算機とも呼ばれ、因数分解、最適化計算に優れるため、同時に話題になっているAI(ディープラーニング)やビックデータなどとの相性が良いことがその一因であることが考えられる。既に商用化された量子コンピュータもあるが、まだまだ課題は多い。量子コンピューターの動作原理である量子ビットには量子性を保ち制御できる仕組みが必要だが、現在それができる仕組みは多くなく、実用化に至るには様々な制約がある。その中で、最も有力だと言われているものが「超伝導回路」による量子ビットである。超伝導回路ではジョセフソン接合を用いて量子もつれ状態を実現しているが、情報の保存時間であるコヒーレンス時間が現在数十マイク秒程度であり、極低温下でしか動作しないなどの課題も多い。北野研究室では、その課題解決に向けて高温超伝導体の固有ジョセフソン接合(IJJ:Intrinsic Josephson Junction)を用いた超伝導量子ビットの実現を目指している。

\section{ジョセフソン接合}
二つの超伝導体が弱く結合した接合をジョセフソン接合と呼ぶ。超伝導状態では、2つの電子が引力相互作用によりクーパー対を形成し、そのクーパー対の状態が巨視的な数重ね合わさり、一つの状態で記述することができるようになる。この状態を記述する関数を、巨視的波動関数と呼ぶ。ジョセフソン接合では2つの(巨視的波動関数の位相が)異なった超伝導体間に薄い絶縁体を挟むなどして弱く結合することにより、その2つの接合にトンネル電流が流れるという現象が見られる。

以下に模式図を示す。2つの超伝導体の位相差を$ \Delta \theta $とすると、そこに流れる超伝導電流は以下のように記述できる。

\[ I = I_c \sin{\Delta \theta } \]




\section{章構成}
あのイーハトーヴォのすきとおった風、夏でも底に冷たさをもつ青いそら、うつくしい森で飾られたモリーオ市、郊外のぎらぎらひかる草の波。
